\documentclass[10pt]{article}
\usepackage{amsmath}
\usepackage{amsfonts}
\usepackage{amssymb}
\usepackage{amsthm}
\usepackage{latexsym}
\usepackage{graphicx}
\usepackage{comment}
\usepackage{tikz}
\usetikzlibrary{arrows.meta,quotes}
\usepackage{hyperref}


\newtheorem{theorem}{Theorem}


% Do not change the following commands.
\addtolength{\oddsidemargin}{-1.1in}
\addtolength{\evensidemargin}{-1.1in}
\addtolength{\textwidth}{2.2in}
\addtolength{\topmargin}{-1in}
\addtolength{\textheight}{1.4in}
% -------------------------------------

\newcommand{\latex}{\LaTeX\xspace}

\renewcommand{\arraystretch}{1.5}

\begin{document}


\title{Homework 1 \\ CCOM3033-002}
\author{Alec Zabel-Mena \\801-16-9720}
\maketitle

\section{Problem 5 (Review question) on page 136.}

\begin{enumerate}
    \item $a = 12*x$

    \item $z= 5*x+14*y+6*k$

    \item $y = x*x*x*x$ or $y=pow(x,2.0)$ (requires cmath.h).

    \item $g = (h+12)/(4*k)$

    \item $c = (a*a*a)/((b*b)*(k*k*k*k))$ or $c = (pow(a,3.0))/(pow(b,2.0)*
        pow(k,4.0))$ (requires cmath.h). 
\end{enumerate}

\section{Problem 25 (Algorithm Workbench) on page 138}
\subsection{Pseudocode for retailCredit.cpp}

We display the pseudocode for retailCredit:

\begin{enumerate}
    \item \textbf{DECLARE:} maxCredit, usedCredit, retailCredit

    \item \textbf{DISPLAY:} ``What is your max credit? ''
        \begin{itemize}
            \item \textbf{INPUT:} maxCredit
        \end{itemize}

    \item \textbf{DISPLAY:} ``What is your used credit? ''
        \begin{itemize}
            \item \textbf{INPUT:} usedCredit
        \end{itemize}

    \item retailCredit=maxCredit-usedCredit

    \item \textbf{DISPLAY:} ``Your retail credit is: '' retailCredit

\end{enumerate}



\end{document}
